\subsection{Filters}
In \texttt{filters.c}, five filters are applied: low-pass, high-pass, derivative, squaring, and moving window integration. In the following section, we present the filter equation, the purpose of the filters, and how we translated the equations to functions in \texttt{C}. 

\subsubsection{Low-pass filter}
If we reconstruct the signal using sine waves (Fourier series), we can choose to filter out sine waves with frequencies too high or too low to reasonably represent a QRS complex. The low-pass filter used has a cutoff frequency of $\approx$ 12 Hz, and is represented by the following difference equation:

\begin{equation}
    y_n = 2y_{n-1} - y_{n-2} + \frac{1}{32} \cdot (x_n - 2x_{n-6} + x_{n-12})
\end{equation}

\subsubsection{High-pass}
The high-pass filter used has a cutoff frequency of $\approx$ 6 Hz, and is represented by the difference equation below. After the data has been through the low-pass and high-pass filters, the remaining signal (in the frequency domain) is centered around 5-20 Hz due to the continuous cutoffs. 
\begin{equation}
    y_n = y_{n-1} -\frac{x_n}{32} + x_{n-16} - x_{n-17} + \frac{x_{n-32}}{32}
\end{equation}

\subsubsection{Derivative}
Now the signal is differentiated, since we are interested in the slope, and most importantly: when the peak (maximum) occurs in the time-domain. 
\begin{equation}
    y_n = \frac{1}{8} \cdot (2x_n + x_{n-1} - x_{n-3} - 2x_{n-4} )
\end{equation}

\subsubsection{Squaring}
Now all values are squared to emphasize large differences in the original data, and to make all values positive.
\begin{equation}
    y_n = x_n^2
\end{equation}

\subsubsection{Moving window integration}
Finally the signal is integrated over the past $N = 30$ signal values. This is achieved by the difference equation below:
\begin{equation}
    y_n = \frac{1}{N} (x_{n- (N-1)}  + x_{n-(N-2)} + ... +  x_n )
\end{equation}

By differentiating, squaring, and then integrating, the filtered data signal is comparable to the original signal; the values represent an amplitude (as a function of time). Each filter causes a delay, e.g. we only obtain the final filtered data point $y_n$ after applying the last filter to the previous 30 data points. See Appendix \eqref{Apx:A} to see how the filters alter the original signal.

\subsubsection{Filters in C}
The function \texttt{filterData} in \texttt{main.c} calls the filter functions in the correct order, by passing memory addresses to arrays, and an int representing the array length. \\
\\
We combined the derivative and squaring filters, and added a function that shifts the array, similar to the \texttt{HackerRank} rotating circular array exercise. After an array has been shifted, the penultimate and ultimate values are identical, so the array is ready for a new value. The following code summarizes how we invoke the filtering in \texttt{main.c}:

\begin{lstlisting}
void filterData(DATA *data){
    shiftArray(data->afterLowPass, ARRAY_LENGTH);
    // applies low pass filter
    data->afterLowPass[ARRAY_LENGTH-1] = lowPassFilter(data->afterLowPass, data->originalArray, ARRAY_LENGTH);
...}
\end{lstlisting}

\texttt{data} is the struct containing all relevant variables: integers, doubles, arrays. The arrays are named after which filter has been applied most recently.
filterData is of type void; it only calls the other filter functions. The filter functions are kept as close to the template as possible, e.g. compare the low pass filter equation with the function \texttt{lowPassFilter}:

\begin{equation}
    y_n = 2 y_{n-1} - y_{n-2} + (x_n + 2x_{n-6}+ x_n{-12})/32
\end{equation}

\begin{lstlisting}
int lowPassFilter(int * y, int * x, int n){
    return 2*y[n-1] - y[n-2] + (x[n] - 2*x[n-6] + x[n-12])/32;
}
\end{lstlisting}

